\RequirePackage[l2tabu, orthodox]{nag}
\documentclass[version=last, pagesize, twoside=off, bibliography=totoc, DIV=calc, fontsize=14pt, a4paper, french, english]{scrartcl}
\input{preamble/packages}
\input{preamble/math_basics}
%Voting and MCDA
\newcommand{\allalts}{\mathscr{A}}
\newcommand{\alts}{A}
\newcommand{\allF}{\mathcal{F}}
\newcommand{\cat}[1]{C_{#1}}

%Voting
\newcommand{\feasalts}{F}
\newcommand{\allvoters}{\mathscr{N}}
\newcommand{\voters}{N}
\newcommand{\allsystems}{\mathcal{G}}
\newcommand{\preflarge}{\boldsymbol{\succeq}^\textbf{r}}%real, complete pref
\newcommand{\pref}{\boldsymbol{\succ}^\textbf{r}}%real, connected pref, strict
\newcommand{\ppreflarge}{\succeq^\text{p}}%partial pref
\newcommand{\ppref}{\succ^\text{p}}%partial pref
\newcommand{\prof}{(\boldsymbol{\succ}^\textbf{r}_i)_{i \in N}}
\newcommand{\profshort}{\boldsymbol{R}^\textbf{r}}
\newcommand{\profs}[1][]{\mathbfcal{R}^\textbf{r}\ifblank{#1}{}{_{#1}}}
\newcommand{\pprof}{(\succ_i^\text{p})_{i \in N}}
\newcommand{\pprofshort}{R^\text{p}}
\newcommand{\allprofs}{\mathcal{L}(\alts)^N}%\mathbfcal{R}
\newcommand{\allpprofs}{\mathcal{O}(\alts)^N}
\newcommand{\linors}{\mathcal{L}(\alts)}
\newcommand{\pors}{\mathcal{O}(\alts)}%partial orders

\newcommand{\allsvs}{\intvl{0, m-1}^N}%all score-vectors
\usepackage{pdfrender}
\newcommand*{\boldgreek}[1]{%
  \textpdfrender{%
    TextRenderingMode=FillStroke,%
    LineWidth=.35pt,%
  }{#1}%
}
%Almost good: $x \text{\textpdfrender{TextRenderingMode=FillStroke,LineWidth=.35pt}{$\succcurlyeq$}\textsuperscript{\textbf{r}}} y$.
%Also: $x \textpdfrender{TextRenderingMode=FillStroke,LineWidth=.35pt}{\succcurlyeq} y$.

%\textpdfrender{TextRenderingMode=FillStroke,LineWidth=.35pt}{$\succcurlyeq$}\textsuperscript{\textbf{r}}
%\newcommand{\prefsv}{\textpdfrender{TextRenderingMode=FillStroke,LineWidth=.35pt}{\succcurlyeq}^\textbf{r}}%pref over score-vectors
\newcommand{\prefsv}{\text{\textpdfrender{TextRenderingMode=FillStroke,LineWidth=.15pt}{$\succcurlyeq$}\textsuperscript{\!\!\!s\textbf{\,r}}}}%pref over score-vectors
\newcommand{\sprefsv}{\mathrel{\boldsymbol{\succ}^{\text{\!\!\!s}\textbf{\,r}}}}%strict pref over score-vectors
\newcommand{\iprefsv}{\mathrel{\boldsymbol{\approx}^{\text{\!\!\!s}\textbf{\,r}}}}%indiff, pref over score-vectors
\newcommand{\pprefsv}{\succcurlyeq^\text{p}}%pref over score-vectors: our approximation

\newcommand{\probprofs}{P^\text{r}}
\newcommand{\probsvs}{P^\text{a}}

%logic atom
%⟼ (long)
\DeclareDocumentCommand{\lato}{ O{\prof} O{\alts} }{[#1 \!⟼\! #2]}
%logic atom in
%↝, \stackrel{\in}{\mapsto}, ➲, ⥹
\newcommand{\tightoverset}[2]{%
  \mathop{#2}\limits^{\vbox to -.5ex{\kern-0.9ex\hbox{$#1$}\vss}}}
\DeclareDocumentCommand{\latoin}{ O{\prof} O{\alpha} }{[#1 \tightoverset{\in}{⟼} #2]}
\newcommand{\alllang}{\mathcal{L}}
\newcommand{\ltru}{\texttt{T}}
\newcommand{\lfal}{\texttt{F}}
\newcommand{\laxiom}[1]{{\texgyreherosfamily{\textsc{#1}}}}

%ARG TH
\newcommand{\AF}{\mathcal{AF}}
\newcommand{\labelling}{\mathcal{L}}
\newcommand{\labin}{\textbf{in}\xspace}
\newcommand{\labout}{\textbf{out}}
\newcommand{\labund}{\textbf{undec}\xspace}
\newcommand{\nonemptyor}[2]{\ifthenelse{\equal{#1}{}}{#2}{#1}}
\newcommand{\gextlab}[2][]{
	\labelling{\mathcal{GE}}_{(#2, \nonemptyor{#1}{\ibeatsr{#2}})}
}
\newcommand{\allargs}{A^*}
\newcommand{\args}{A}
\newcommand{\ar}{a}
\newcommand{\ext}{\mathcal{E}}

%MCDA+Arg
\newcommand{\dm}{d}
\newcommand{\ileadsto}{\rightcurvedarrow}
\newcommand{\mleadsto}[1][\eta]{\rightcurvedarrow_{#1}}
\newcommand{\ibeats}{\vartriangleright}
\newcommand{\mbeats}[1][\eta]{\vartriangleright_{#1}}

%MISC
\newcommand{\lequiv}{\Vvdash}
\newcommand{\weightst}{W^{\,t}}

%MCDA classical
\newcommand{\crits}{\mathcal{J}}

%Sorting
\newcommand{\cats}{\mathcal{C}}
\newcommand{\catssubsets}{2^\cats}
\newcommand{\catgg}{\vartriangleright}
\newcommand{\catll}{\vartriangleleft}
\newcommand{\catleq}{\trianglelefteq}
\newcommand{\catgeq}{\trianglerighteq}
\newcommand{\alttoc}[2][x]{(#1 \xrightarrow{} #2)}
\newcommand{\alttocat}[3]{(#2 \xrightarrow{#1} #3)}
\newcommand{\alttoI}{(x \xrightarrow{} \left[\underline{C_x}, \overline{C_x}\right])}
\newcommand{\alttocatdm}[3][t]{\left(#2 \thinspace \raisebox{-3pt}{$\xrightarrow{#1}$}\thinspace #3\right)}
\newcommand{\alttocatatleast}[2]{\left(#1 \thinspace \raisebox{-3pt}{$\xrightarrow[]{≥}$}\thinspace #2\right)}
\newcommand{\alttocatatmost}[2]{\left(#1 \thinspace \raisebox{-3pt}{$\xrightarrow[]{≤}$}\thinspace #2\right)}

\input{preamble/redac}
\input{preamble/draw}
\input{preamble/acronyms}

\begin{document}
\title{Simultaneous elicitation for voting rules}
\author{Olivier Cailloux}
\author{Stefano Moretti}
\affil{Université Paris-Dauphine, 
PSL Research University, 
CNRS, LAMSADE, 
75016 PARIS, FRANCE
}
\author{Paolo Viappiani\footnote{Authors are currently listed according to alphabetical order.}}
\affil{CNRS-LIP6 laboratory - Université Pierre-et-Marie-Curie (UPMC),
4, Place Jussieu 75252 Paris, France
}
\makeatletter
	\hypersetup{
		pdfsubject={Social choice},
		pdfkeywords={elicitation, active learning}
	}
\makeatother
\maketitle

\section{Introduction}
In some situations a committee may desire to select winning alternatives using the preferences, over those alternatives, of multiple individuals (called voters). For example, a jury may desire to use the opinions of multiple reviewers to elect the best papers among some given set of papers. We are interested in situations where the committee wants to express preferences on the voting rule and voters express preferences about the order of the alternatives. We assume the committee expresses only one voice: preference on the voting rule is consensual. Voters may have differing opinions however. The voting rule serves to aggregate those different opinions.

We assume the preferences on the voting rule (of the committee) and over alternatives (of the voters) are unknown and obtaining information about them have a cost. This could be because we care about privacy (knowing as few as necessary about other’s preferences might be considered good), because of transmission cost (?), because of cognitive cost or other resources required for someone to make up her preferences, including, in the case of the committee, the cost of coming to an agreement. Hence, we are interested in querying as few as possible.

We are interested in solving a given problem, meaning, finding (or getting close to) the set of winning alternatives, given a set of alternatives, a committee and a set of individuals with (well-defined but unknown) preferences. As a bonus however, we also obtain some reusable information about the voting rules the committee appreciates, which might be useful if the procedure has to be applied repeatedly with the same committee.

Here are some notations. In those notations, the relations in bold suffixed with the letter \textbf{r} indicate the “real” relations, which we want to distinguish from the approximations we will build (suffixed with the letter $p$ to indicate partial knowledge).
\begin{description}[font=\normalfont, leftmargin=!, labelwidth=\widthof{$F: \allprofs \rightarrow \powersetz{\alts}$}]
	\item[$N ≠ \emptyset$] the (finite) set of voters
	\item[$\alts$] the (finite) set of alternatives, $\card{\alts} = m$
	\item[$\linors \subseteq \powerset{\alts × \alts}$] the linear orders (connected, transitive, asymmetric) on $\alts$
	\item[$\pors \subseteq \powerset{\alts × \alts}$] the partial orders (transitive, asymmetric) on $\alts$
	\item[$\pref_i \in \linors$] the preference of voter $i$ (called a complete preference)
	\item[$\ppref_i \in \pors$] a partial preference associated to $i$, representing our knowledge of $\pref_i$
	\item[$\prof = \profshort$] a profile of complete preferences, called a complete profile
	\item[$\pprof = \pprofshort$] a profile of partial preferences, called a partial profile
	\item[$\allprofs$] the set of all complete profiles
	\item[$\allpprofs$] the set of all partial profiles
	\item[$\powersetz{\alts}$] the set of all subsets of $\alts$ minus the emptyset
	\item[$F: \allprofs \rightarrow \powersetz{\alts}$] a voting rule (associates each complete profile to a non-empty subset of winning alternatives)
\end{description}

Throughout the article we consider defined a voting rule $G$ that is used to specify the preferences on the voting rule of the committee: given a complete profile $\prof$, the committee will select a winner among $G(\prof)$ (the tie-breaking being done in a way not described by $G$). We consider the committee is able to express some of its preferences under the form of constraints on $G$, such as: if $G$ selects such alternative given such profile, then it may not select such alternative given such profile.
When talking about some generic voting rule we use the letter $F$. 

We can extend the domain of a voting rule to all partial profiles instead of only complete profiles. Given a rule $F$, define the robust equivalent rule as $F^\text{robust}: \allpprofs → \powersetz{\alts}$ which selects all possible winners given the partial profile, thus, an alternative is in $F^\text{robust}(\pprof)$ if there exists a complete profile $\prof$ that is a completion of $\pprof$ (meaning that for each $i$, $\pref_i$ is a completion of $\ppref_i$) for which $F(\prof)$ selects that alternative.

\subsection{Scoring-vectors}
Define a scoring-vector $x$ as an element of $\allsvs$, a mapping of voters to scores, where a score is an integer in $\intvl{0, m-1}$. (This notation designates intervals in the natural numbers throughout this article.) As $m$ is less than $10$ in our examples, we write our examples of scoring-vector as concatenated numbers, for example $43$ denotes a scoring-vector mapping the first voter to score $4$ and the second voter to score $3$.

Given a complete profile $\profshort = \prof$, we can associate each alternative $a$ to its scoring-vector $χ_{\profshort}(a) \in \allsvs$, by giving, for each voter $i \in N$, as many points to $a$ as the number of alternatives it is preferred to in $\pref_i$. Thus, $χ_{\profshort}$ is the function in $(\allsvs)^{A}$ that associates alternatives to scoring-vectors on the basis of the profile $\profshort$. 
We denote a scoring-vector as $x_a$ instead of $χ_{\profshort}(a)$ when the profile is clear from the context. 
For example, if $i$ puts $a$ last in her ranking $\pref_i$, $x_a$ maps $i$ to $0$. 
This is illustrated in \cref{fig:sv}.
\begin{figure}[t]
	\centering
	\begin{tikzpicture}
		\tikzset{prof matrix/.style={
			matrix, column sep=3mm, row sep=2mm
		}}
		\tikzset{rank-vector/.style={
			draw, rectangle, inner sep=0, outer sep=1mm
		}}
		
		\path node[prof matrix] (profile) {
			\path node {$\pref_1$};&
			\path node {$\pref_2$};
			\\
			\path node {$a$};&
			\path node {$a$};
			\\
			\path node {$b$};&
			\path node {$c$};
			\\
			\path node {$c$};&
			\path node {$b$};
			\\
		};
		\path[draw, decorate, decoration={brace, mirror}] (profile.south west) -- (profile.south east);
		\path ($(profile.south west)!.5!(profile.south east)$) ++ (0, -5mm) node {$\prof \in \linors^N$};
		
		\path (profile.north east) ++ (1.5cm, 0) node[prof matrix, anchor=north west] (rank-profile) {
			&
			\path node (header start) {1};&
			\path node (header end) {2};
			\\
			\path node (alts start) {$a$};&
			\path node (rv1 start) {2};&
			\path node (rv1 end) {2};
			\\
			\path node {$b$};&
			\path node (rv2 start) {1};&
			\path node (rv2 end) {0};
			\\
			\path node (alts end) {$c$};&
			\path node (rv3 start) {0};&
			\path node (rv3 end) {1};
			\\
		};
		\path node[draw, ellipse, dotted, inner sep=0, fit=(header start.north west) (header end.south east)] (N) {};
		\path (N.north) node[anchor=south, inner sep=1mm] {$N$};
		\path node[draw, ellipse, dotted, inner sep=0, fit=(alts start.north west) (alts end.south east)] (A) {};
		\path (A.west) node[anchor=east, inner sep=1mm] {$A$};
		\foreach \i/\a in {1/a, 2/b, 3/c} {
			\path node[rank-vector, fit=(rv\i\space start.north west) (rv\i\space end.south east)] (rv\i) {};
			\path[<-, draw] (rv\i.east) -- ++(0.3cm, 0) node[anchor=west] {$x_\a \in \allsvs$};
		}
	\end{tikzpicture}
	\caption{A complete profile and the corresponding scoring-vectors (adapted from \citet{cailloux_eliciting_2014})}
	\label{fig:sv}
\end{figure}

We say that a function in $(\allsvs)^{A}$ is a scoring-vectors profile iff it corresponds to some complete profile, thus, iff, for each voter, it gives different scores to different alternatives (using the presentation of \cref{fig:sv}, this forbids to repeat a number in a given column).
We can now view each complete profile as equivalent to its corresponding scoring-vectors profile: the just described mapping $\set{(\profshort, χ_{\profshort}), \profshort \in \allprofs}$ is a bijection.
%We will thus use interchangeably $\profshort$ and $χ_{\profshort}$ when it causes no confusion, for example, we consider $F(χ_{\profshort})$ is defined (and equal to $F(\profshort)$).
Abusing notation, we say that a scoring-vector $x$ wins in $F(\profshort)$, and write $x \in F(\profshort)$, to mean that $x \in χ_{\profshort}(F(\profshort))$. 
We say that a complete profile $\profshort$ contains a scoring-vector iff some alternative is associated to this scoring-vector in $χ_{\profshort}$, that is, iff $x \in χ_{\profshort}(\allalts)$. 
Finally, we say that two scoring-vectors $x, y \in \allsvs$ are incompatible iff they can’t be together in a profile, that is, iff $\exists i \in N \suchthat x(i) = y(i)$.
%We may also consider any voting rule $F$ interchangeably with its scoring-vectors voting rule counterpart $F^\text{s}$, that selects winning scoring-vectors given scoring-vectors profiles instead of winning alternatives given profiles. Indeed, given $F$, we can define $F^\text{s}(χ_{\profshort}) = χ_{\profshort}(F(\profshort))$, and this mapping is a bijection. 

\subsection{Neutrality}
Some voting rules are neutral, meaning that they do not consider the identity of the alternatives, thus, that they consider only the preferences to select winners. 
%Formally, let $\sigma$ denote a permutation over the alternatives $\allalts$, that is, a bijection from $\allalts$ to $\allalts$, and let $\boldsymbol{\succ}^\textbf{r}_{i, \sigma} = \sigma \circ \boldsymbol{\succ}^\textbf{r}_i \circ \sigma^{-1}$ denote the preference after permuting the alternatives. For example, if $\sigma$ maps $x$ to $w$ and $y$ to $z$ and $\boldsymbol{\succ}^\textbf{r}_i$ contains $(x, y)$, then $\boldsymbol{\succ}^\textbf{r}_{i, \sigma}$ contains $(y, z)$. 

Define a neutral profile as a profile that does not include the identity of the alternatives. Define a neutral scoring-vectors profile as any set of $m$ scoring-vectors that are pairwise compatible. Note that there are $m!$ scoring-vectors profiles for each neutral scoring-vectors profile. 
%Assume $N ≠ \emptyset$, and let $i_0 \in N$ denote one of the voters in $N$.
%A neutral profile $\neutrprof$ maps each voter except $i_0$ to its preference, thus, considers the preference of $i_0$ fixed. There is a bijection between neutral scoring-vectors profiles and neutral profiles.

When a voting rule $F$ is neutral, it may be considered as selecting winning scoring-vectors given neutral scoring-vectors profiles.

\section{Querying the preference on the voting rule through scoring-vectors}
We now want to define a way of querying whether, intuitively, $G$ (and thus, the committee) “considers” $x$ as a better scoring-vector than $y$. We will assume we can obtain information about this ordering (by interrogating the committee) and use this to enrich our knowledge of $G$.

Consider a binary relation $\prefsv$ over the scoring-vectors. Define $\sprefsv$ as its asymmetric part and $\iprefsv$ as its symmetric part. We ask queries of the form $x \prefsv y$, and we assume that the answers of the committee to those queries satisfy the following properties. First, when the committee tells us that $x \sprefsv y$, it implies that whenever both $x$ and $y$ are present in a profile $\profshort$, $y$ does not win in $G(\profshort)$. Second, when the committee tells us that $x \iprefsv y$, it implies that $x$ is selected by $G(\profshort)$ whenever $y$ is selected by $G(\profshort)$, for all profiles $(\profshort)$ where both $x$ and $y$ are present. Third, $\prefsv$ is a preorder (a reflexive and transitive binary relation).

Let $\profs[xy] = \set{\profshort \suchthat \{x, y\} \subseteq χ_{\profshort}(\allalts)}$ denote the set of all complete profiles that contain the scoring-vectors $x$ and $y$. Given a rule $F$, $x \in \bigcup F(\profs[xy])$ holds whenever $F(\profshort)$ selects $x$ as one of the winners for some complete profile $\profshort$ that contains $x$ and $y$.

Our hypothesis is thus (H1) that $x \sprefsv y$ implies $y \notin \bigcup G(\profs[xy])$, (H2) that $x \iprefsv y$ implies $\forall \profshort \in \profs[xy]: x \in G(\profshort) ⇔ y \in G(\profshort)$, and (H3) that $\prefsv = \sprefsv ∪ \iprefsv$ is a preorder. (\Cref{sec:defSprefsv} analyzes alternative hypotheses.)

The transitivity of $\prefsv$ makes it a useful basis for queries: a few questions can bring “free” information, by computing the transitive closure of our knowledge so far.

Assume we are able to obtain a strict partial order (a transitive and asymmetric relation) $\psprefsv \subseteq \sprefsv$. Define our approximation of $G$, $F_{\psprefsv}$, as the rule that selects the maximal elements (according to $\psprefsv$) of the given complete profile: $x$ wins in $F_{\psprefsv}(\profshort)$ iff $x$ is in $\profshort$ and there is no $y$ in $\profshort$ such that $y \psprefsv x$. The rule $F_{\psprefsv}$ gives us information about the real rule $G$, in the sense that any real winner is among the winners selected by $F_{\psprefsv}$.
\begin{fact}
	\label{thm:approx}
	Given $\psprefsv \subseteq \sprefsv$, $\forall \profshort: G(\profshort) \subseteq F_{\psprefsv}(\profshort)$.
\end{fact}
\begin{proof}
	Consider $\profshort$ and assume $y$ does not win in $F_{\psprefsv}(\profshort)$. We show that $y$ does not win in $G(\profshort)$. We know that $y$ is dominated by some $x$ in $\profshort$, thus $x \psprefsv y$ and both $x$ and $y$ are in $\profshort$. Because $x \psprefsv y$, $x \sprefsv y$, thus $y$ does not win in $G(\profshort)$.
\end{proof}

By collecting the answers to our queries in a preorder $\pprefsv$, we keep getting closer to $\prefsv$ and thus can approximate better and better the rule $G$ (without necessarily being able to reach it, depending on the richness of $\prefsv$).
\begin{fact}
	Assume $R \subseteq \sprefsv$ and $S \subseteq \iprefsv$, with $S$ reflexive and symmetric. Define $\pprefsv = (R ∪ S)^T$, and $\psprefsv$ and $\piprefsv$ as the asymmetric and symmetric parts of $\pprefsv$. Then, $\pprefsv \subseteq \prefsv$, $\psprefsv \subseteq \sprefsv$ and $\piprefsv \subseteq \iprefsv$.
\end{fact}
\begin{proof}
	Observe that $R$ is asymmetric, as $R \subseteq \sprefsv$, and that $\pprefsv$ is a preorder.

	Consider $x \pprefsv y$. Thus, $x (R ∪ S)^T y$. Thus, $x \prefsv y$. Now we show that exactly two disjoint cases are possible: either $x \piprefsv y$ and $x \iprefsv y$, or $x \psprefsv y$ and $x \sprefsv y$.
	
	If from $x$ we can reach $y$ without using $R$, thus using only $S$, and because $S$ is symmetric, then from $y$ we can also reach $x$, thus $y S^T x$, hence $y \pprefsv x$ and $y \prefsv x$, thus $x \piprefsv y$ and $x \iprefsv y$. 

	Otherwise, we must use $R$ to reach $y$. Let $x_1 \mathrel{R} x_2$ be on the way from $x$ to $y$. Thus, $x \pprefsv x_1 \mathrel{R} x_2 \pprefsv y$, thus $x \prefsv x_1 \sprefsv x_2 \prefsv y$, thus $¬(x_2 \prefsv x_1)$, thus $¬(y \prefsv x)$ (otherwise, there is a link from $x_2$ to $y$ to $x$ to $x_1$), thus $¬(y \pprefsv x)$, thus $x \psprefsv y$ and $x \sprefsv y$.
\end{proof}

\section{Characterisation of scoring-vector rules}
A rule $F$ is scoring-vector-based iff it can be described as maximizing a strict partial order over scoring-vectors, or formally, iff for some strict partial order $\sprefsv$ over $\allsvs$, $F = F_{\sprefsv}$. A sub-class of this class is the class of weak-order based rules, corresponding to the case where $\sprefsv$ is a connected relation (thus, orders any two different scoring-vectors).

\citet{cailloux_eliciting_2014} gave some preliminary analysis of those classes of rules.

\commentOC{Il pourrait être intéressant de poursuivre cette analyse. Et peut-être même utile pour une élicitation efficace ?}

\section{Elicitation of the preference on the voting rule only}
We assume we can observe $\sprefsv$ and $\iprefsv$: provided with a pair $x, y$, we can know whether $x \sprefsv y$, $y \sprefsv x$, $x \iprefsv y$, or none of those.

We start by assuming that $\prefsv$ is a weak-order (a connected preorder), thus “none of those” does not happen under this assumption. This implies that $G$ is neutral.

\begin{fact}
	If $\prefsv$ is a weak-order, then $G$ is neutral.
\end{fact}
\begin{proof}
	We show that $G = F_{\sprefsv}$, then, we show that $F_{>}$ is neutral for any strict partial order $>$ on the scoring-vectors. 
	
	From \cref{thm:approx}, $\forall \profshort: G(\profshort) \subseteq F_{\sprefsv}(\profshort)$. Because $\prefsv$ is complete, all scoring-vectors winning in $F_{\sprefsv}(\profshort)$ (all maximal elements of $\prefsv$) are equivalent from the point of view of $\prefsv$. Thus, by our hypothesis (H2), $G$ either selects none of them, or all. Because $\emptyset ≠ G(\profshort) \subseteq F_{\sprefsv}(\profshort)$, $G$ selects at least one of them, thus, it selects all of them, and $G(\profshort) = F_{\sprefsv}(\profshort)$.
	
	Given any strict partial order $>$ on the scoring-vectors (thus, including $\prefsv$), $F_>$ uses no information pertaining to the identity of the alternatives, thus, is neutral.
\end{proof}

We call a committee query a question in the form of a pair of scoring-vectors, to which we obtain as answer: $x \sprefsv y$, $y \sprefsv x$, or $x \iprefsv y$. We keep track of those answers by way of building iteratively a preorder $\pprefsv$.

We assume here that $\prof$ is not defined: we are interested in eliciting the preference on the voting rule of the committee, meaning, the function $G$, so that it can then be applied to any profile when known.

We consider a probability distribution $P^r$ over the complete profiles is known: $\sum_{\profshort \in \allprofs} \probprofs(\profshort) = 1$.

Define the badness $V_{\pprefsv}$ of our current approximation $\pprefsv$ as follows: $V_{\pprefsv} = \sum_{\profshort \in \allprofs} \probprofs(\profshort) \card{F_{\psprefsv}(\profshort)}$.

As further simplification hypotheses, we assume for now an impartial culture, thus all profiles are equally likely ($\probprofs$ is uniform), thus, $V_{\pprefsv} = \frac{1}{\card{\allprofs}} \card{F_{\psprefsv}(\profshort)}$. %and we consider that the real rule $G$ is resolute, thus $\card{G(\profshort)}=1, \forall \profshort \in \allprofs$, thus, $V_{\pprefsv} = 1 ⇔ F_{\psprefsv} = G$.

Define $\probsvs_{\pprefsv}(x \sprefsv y)$ as our estimation, given our knowledge $\pprefsv$, of the probability that the committee answers $x \sprefsv y$ (this depends on some probability estimate over the set of possible weak-orders over scoring-vectors, probability estimate which in turn depends on $\pprefsv$, but which we do not need to make explicit for now). We define similarly $\probsvs_{\pprefsv}(x \iprefsv y)$.

We define the badness of asking a question of comparison of $x$ and $y$ as $b(x, y) = \probsvs_{\pprefsv}(x \sprefsv y) V_{\pprefsv ∪ \{(x, y)\}} + \probsvs_{\pprefsv}(y \sprefsv x) V_{\pprefsv ∪ \{(y, x)\}} + \probsvs_{\pprefsv}(x \iprefsv y) V_{\pprefsv ∪ \{(x, y), (y, x)\}}$. The lower, the better.

We would like to find out how to compute $V_{\pprefsv}$, without having to iterate on the neutral profiles: when $m=6, n=6$, $\card{\neutrprofs} = (6!)^5 > 10^{14}$.

Given two disjoint sets $B, C \subseteq \allsvs$ such that all scoring-vectors from $B$ are pairwise compatible, let $\#_{B, ¬C}$ denote the number of neutral profiles that contain all scoring-vectors from $B$ and no scoring-vectors from $C$.
Note that if some scoring-vector in $C$ is not pairwise compatible with some scoring-vector in $B$, then that scoring-vector from $C$ can be removed from $C$ without changing the number of profiles.
Define $\#_B = \#_{B, ¬\emptyset}$.
Define $b = \card{B}, c = \card{C}$. 

Counting all profiles and not just the compatible ones, we obtain: $\#_B^\text{all} = \binom{m^n - b}{m-b}$. This is because we have to choose $m-b$ scoring-vectors to complete the profile (once the ones from $B$ have been fixed), and those have to be chosen from all the remaining possibilities; starting with a total of $m^n$ scoring-vectors.

Observe that $\#_B = [(m-b)!]^{n-1}$.

Let $\powersetz{C}^\text{compat}$ denote the set of non-empty subsets of $C$ that contain only pairwise compatible scoring-vectors. 
\begin{fact}
	$\#_{B, ¬C} = \#_B − (\sum_{K \in \powersetz{C}^\text{compat}} \#_{B ∪ K, ¬(C \setminus K)})$.
\end{fact}
\begin{proof}
	When counting all profiles containing $B$ (that is, $\#_B$), we’ve counted too many. We have to remove the ones that contain all scoring-vectors from $B$ and some elements from $C$, equivalently, we have to remove the ones that contain all scoring-vectors from $B$ and exactly the scoring-vectors in $K$ for some subset $K$ of $C$, and none others from $C$.
\end{proof}
Note that equivalently, assuming we start with $C$ containing only scoring-vectors that are pairwise compatible with each scoring-vector in $B$, we obtain $\#_{B, ¬C} = \#_B − (\sum_{K \in \powersetz{C}^\text{compat}} \#_{B ∪ K, ¬\mathcal{C}(C, K)})$, where $\mathcal{C}(C, K)$ denotes the scoring-vectors from $C$ that are pairwise compatible with each vector in $K$.

\begin{fact}
	When $B ∪ C$ contain only pairwise compatible scoring-vectors, $\#_{B, ¬C} = \#_b − (\sum_{1 ≤ k ≤ c} \binom{c}{k} \#_{b+k, ¬(c-k)}) = \#_b − (\sum_{0 ≤ d < c} \binom{c}{d} \#_{b+c-d, ¬d})$.
\end{fact}
\begin{proof}
	The first equality follows from the previous fact, the last equality uses $d = c-k$.
\end{proof}
Observe that the number depends only on the cardinals of $B$ and $C$. 

(?) Also, writing $\#_{B, ¬C}^m$ for the case where the number of alternatives is $m$, and assuming that $C$ contains only scoring-vectors that are pairwise compatible with each scoring-vector in $B$, $\#_{B, ¬C}^m = \#_{0, ¬C}^{m-b}$.

\subsection{Draft}
Consider $B ∪ C$ contains only pairwise compatible scoring-vectors. $\#_{0, ¬c}$: order the vectors in $C$ by those starting with the score $1$, then $2$, and so on (rename the scores if necessary). Choose among $(m^{n-1} - 1)$ to fix the first scoring-vector: because there’s $m$ figures to choose from for each voter except the first one, as the scoring-vector starts with one (to fix the ordering of s-vs in the profile), and then there’s the scoring-vector from $C$ that starts with one that we can’t choose, thus substracting one. Then choose among $(m-1)^{n-1}-1$ to fix the second one, and so on for $c$ times, then choose among $(m-x)^{n-1}$ because we don’t have to substract one any more. But this is incorrect because we could have chosen as first scoring-vector one that conflicts with members of $C$, which augments the remaining possibilities (no need to substract one).

Let’s decompose the preorder $\pprefsv$ into equivalence classes, one per “level”, where the level 1 contains the maximal scoring-vectors of $\pprefsv$, the level 2 contains the next best scoring-vectors, and so on. Write $z$ the last level of $\pprefsv$. Write $\alpha(l)$ the number of scoring-vectors at level $l$, $\beta(l)$ the number of scoring-vectors that are better than those at level $l$, thus $\beta(l) = \sum_{1 ≤ k < l} \alpha(k)$. Now $V_{\pprefsv} = \sum_{1 ≤ l ≤ z}\sum_{1 ≤ k ≤ \alpha(l)} k \#_{k, ¬(\alpha(l) - k + \beta(l))}$. Or, more likely, $V_{\pprefsv} = \sum_{1 ≤ l ≤ z}\sum_{1 ≤ k ≤ \alpha(l)} \binom{\alpha(l)}{k} k \#_{k, ¬(\alpha(l) - k + \beta(l))}$. \emph{Except} this does not account for the cases where some of these rank-vectors are incompatible.

\section{Simultaneous elicitation}
We call an individual query a question in the form of a pair of alternatives and a voter $i \in N$. We assume the answer is true, thus, conforms to the real relation $\pref_i$. We keep track of those answers by building iteratively a set of relations $\pprof$.

At each point in time we know $\pprefsv \subseteq \prefsv$ and $\ppref_i \subseteq \pref_i, \forall i \in N$, and hence, $F(\prof) \subseteq F_{\pprefsv}(\pprof)$.

We want to ask questions in a way that we get close to the real winners, $W = F(\prof)$, using few queries.

\bibliography{elic.bib}

\appendix
\section{About the definition of the relation over scoring-vectors}
\label{sec:defSprefsv}
It might be conceptually useful to make clear the minimal properties we need for $\sprefsv$, and to determine the minimal required hypothesis about the relation between what the committee says and $G$. Or to obtain an “iff” definition for $\sprefsv$ relating it to $G$.

Define binary relations over $\allsvs$ as follows.
\begin{itemize}
	\item $x B_> y$ iff $y \notin \bigcup G(\profs[xy]) ∧ x \in \bigcup G(\profs[xy])$
	\item $x B_\text{b} y$ iff $y \in \bigcup G(\profs[xy]) ∧ x \in \bigcup G(\profs[xy])$ (the letter b stands for “both”)
	\item $x B_\text{n} y$ iff $y \notin \bigcup G(\profs[xy]) ∧ x \notin \bigcup G(\profs[xy])$ (the letter n stands for “neither”)
	\item $x B_\text{w} y$ iff $x B_> y ∨ x B_\text{b} y$ iff $x \in \bigcup G(\profs[xy])$ (the letter w stands for “weak”)
	\item $x B^T y$, for any relation $B$, equals the transitive closure of $B$.
\end{itemize}

Note that we can’t suppose in general that $x \sprefsv y ⇔ y \notin \bigcup G(\profs[xy])$, as then $x \sprefsv y \sprefsv x$ for incompatible scoring-vectors. Also, we can’t suppose in general that $\sprefsv = B_>$. First, because $B_>$ may not be a strict partial order, and taking its transitive closure may not solve the problem: $G$ could be such that $(x = 21) B_> (y = 12) B_> (z = 20) B_> (w = 02) B_> (x = 12)$. The second inadequacy of $B_>$ is that sometimes, $x \sprefsv y$ seems intuitively meaningful even though $x$ and $y$ can’t be together in a profile: consider $44$ and $41$. In that case, a plausible interpretation of the intuition is that $44$ is better than a third scoring-vector, itself better than $41$ (and that no cycle happen): consider $44 B_> 33 B_> 41$.

We obtain the following possible definition relating the relation $\sprefsv$ and the rule $G$: $x \sprefsv y$ iff [($x B_> y$) or ($x B_>^T y$ and $x B_n y$)] and not $y B_\text{w}^T x$.

\commentOC{On pourrait peut-être explorer la relation entre $G$ et $\prefsv$, voir la remarque ci-dessous.}

Remark that there are two properties in particular that we need, about $\prefsv$ and the relationship between $\prefsv$ and $G$: $\prefsv$ is a preorder, and $G \subseteq G_{\prefsv}$. Our definition of $\prefsv$, given $G$, permits to obtain the properties we want. However, we could relax our demands about $\prefsv$ and still obtain the desired properties. Consider as an example $m = 3$, one voter, and the pareto-compatible rule, which thus selects always the best ranked alternative, equivalently, which selects the scoring-vector $3$ as sole winner. Observe that $G$ determines according to our definitions a preorder where the scoring-vector $2$ is indifferent to $1$ (because both are dominated by $3$, and this is all that matters to $G$). Assume the committee tells us on the contrary that $2 \sprefsv 1$. We can still obtain the properties we want, although it contradicts our definition of $\sprefsv$.

\end{document}

