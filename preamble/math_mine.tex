%Voting and MCDA
\newcommand{\allalts}{\mathscr{A}}
\newcommand{\alts}{A}
\newcommand{\allF}{\mathcal{F}}
\newcommand{\cat}[1]{C_{#1}}

%Voting
\newcommand{\feasalts}{F}
\newcommand{\allvoters}{\mathscr{N}}
\newcommand{\voters}{N}
\newcommand{\allsystems}{\mathcal{G}}
\newcommand{\preflarge}{\boldsymbol{\succeq}^\textbf{r}}%real, complete pref
\newcommand{\pref}{\boldsymbol{\succ}^\textbf{r}}%real, connected pref, strict
\newcommand{\ppreflarge}{\succeq^\text{p}}%partial pref
\newcommand{\ppref}{\succ^\text{p}}%partial pref
\newcommand{\prof}{(\boldsymbol{\succ}^\textbf{r}_i)_{i \in N}}
\newcommand{\profshort}{\boldsymbol{R}^\textbf{r}}
\newcommand{\profs}[1][]{\mathbfcal{R}^\textbf{r}\ifblank{#1}{}{_{#1}}}
\newcommand{\pprof}{(\succ_i^\text{p})_{i \in N}}
\newcommand{\pprofshort}{R^\text{p}}
\newcommand{\allprofs}{\mathcal{L}(\alts)^N}%\mathbfcal{R}
\newcommand{\anonprofs}{\mathcal{L}(\alts)^{N\setminus\set{i_0}}}%\mathbfcal{R}
\newcommand{\allpprofs}{\mathcal{O}(\alts)^N}
\newcommand{\linors}{\mathcal{L}(\alts)}
\newcommand{\pors}{\mathcal{O}(\alts)}%partial orders

\newcommand{\allsvs}{\intvl{0, m-1}^N}%all score-vectors
\usepackage{pdfrender}
\newcommand*{\boldgreek}[1]{%
  \textpdfrender{%
    TextRenderingMode=FillStroke,%
    LineWidth=.35pt,%
  }{#1}%
}
%Almost good: $x \text{\textpdfrender{TextRenderingMode=FillStroke,LineWidth=.35pt}{$\succcurlyeq$}\textsuperscript{\textbf{r}}} y$.
%Also: $x \textpdfrender{TextRenderingMode=FillStroke,LineWidth=.35pt}{\succcurlyeq} y$.

%\textpdfrender{TextRenderingMode=FillStroke,LineWidth=.35pt}{$\succcurlyeq$}\textsuperscript{\textbf{r}}
%\newcommand{\prefsv}{\textpdfrender{TextRenderingMode=FillStroke,LineWidth=.35pt}{\succcurlyeq}^\textbf{r}}%pref over score-vectors
\newcommand{\prefsv}{\mathrel{\text{\textpdfrender{TextRenderingMode=FillStroke,LineWidth=.15pt}{$\succcurlyeq$}\textsuperscript{\!\!\!s\textbf{\,r}}}}}%pref over score-vectors
\newcommand{\sprefsv}{\mathrel{\boldsymbol{\succ}^{\text{\!\!\!s}\textbf{\,r}}}}%strict pref over score-vectors
\newcommand{\iprefsv}{\mathrel{\boldsymbol{\sim}^{\text{\!\!\!s}\textbf{\,r}}}}%indiff, pref over score-vectors
\newcommand{\pprefsv}{\succcurlyeq^\text{\!\!\!s\,p}}%pref over score-vectors: our approximation
\newcommand{\psprefsv}{\succ^\text{\!\!\!s\,p}}%strict version
\newcommand{\piprefsv}{\sim^\text{\!\!\!s\,p}}

\newcommand{\probprofs}{P^\text{r}}
\newcommand{\probsvs}{P^\text{a}}

%logic atom
%⟼ (long)
\DeclareDocumentCommand{\lato}{ O{\prof} O{\alts} }{[#1 \!⟼\! #2]}
%logic atom in
%↝, \stackrel{\in}{\mapsto}, ➲, ⥹
\newcommand{\tightoverset}[2]{%
  \mathop{#2}\limits^{\vbox to -.5ex{\kern-0.9ex\hbox{$#1$}\vss}}}
\DeclareDocumentCommand{\latoin}{ O{\prof} O{\alpha} }{[#1 \tightoverset{\in}{⟼} #2]}
\newcommand{\alllang}{\mathcal{L}}
\newcommand{\ltru}{\texttt{T}}
\newcommand{\lfal}{\texttt{F}}
\newcommand{\laxiom}[1]{{\texgyreherosfamily{\textsc{#1}}}}

%ARG TH
\newcommand{\AF}{\mathcal{AF}}
\newcommand{\labelling}{\mathcal{L}}
\newcommand{\labin}{\textbf{in}\xspace}
\newcommand{\labout}{\textbf{out}}
\newcommand{\labund}{\textbf{undec}\xspace}
\newcommand{\nonemptyor}[2]{\ifthenelse{\equal{#1}{}}{#2}{#1}}
\newcommand{\gextlab}[2][]{
	\labelling{\mathcal{GE}}_{(#2, \nonemptyor{#1}{\ibeatsr{#2}})}
}
\newcommand{\allargs}{A^*}
\newcommand{\args}{A}
\newcommand{\ar}{a}
\newcommand{\ext}{\mathcal{E}}

%MCDA+Arg
\newcommand{\dm}{d}
\newcommand{\ileadsto}{\rightcurvedarrow}
\newcommand{\mleadsto}[1][\eta]{\rightcurvedarrow_{#1}}
\newcommand{\ibeats}{\vartriangleright}
\newcommand{\mbeats}[1][\eta]{\vartriangleright_{#1}}

%MISC
\newcommand{\lequiv}{\Vvdash}
\newcommand{\weightst}{W^{\,t}}

%MCDA classical
\newcommand{\crits}{\mathcal{J}}

%Sorting
\newcommand{\cats}{\mathcal{C}}
\newcommand{\catssubsets}{2^\cats}
\newcommand{\catgg}{\vartriangleright}
\newcommand{\catll}{\vartriangleleft}
\newcommand{\catleq}{\trianglelefteq}
\newcommand{\catgeq}{\trianglerighteq}
\newcommand{\alttoc}[2][x]{(#1 \xrightarrow{} #2)}
\newcommand{\alttocat}[3]{(#2 \xrightarrow{#1} #3)}
\newcommand{\alttoI}{(x \xrightarrow{} \left[\underline{C_x}, \overline{C_x}\right])}
\newcommand{\alttocatdm}[3][t]{\left(#2 \thinspace \raisebox{-3pt}{$\xrightarrow{#1}$}\thinspace #3\right)}
\newcommand{\alttocatatleast}[2]{\left(#1 \thinspace \raisebox{-3pt}{$\xrightarrow[]{≥}$}\thinspace #2\right)}
\newcommand{\alttocatatmost}[2]{\left(#1 \thinspace \raisebox{-3pt}{$\xrightarrow[]{≤}$}\thinspace #2\right)}
